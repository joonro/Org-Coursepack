% Intended LaTeX compiler: pdflatex
\documentclass[10pt,article]{article}
\usepackage[utf8]{inputenc}
\usepackage[T1]{fontenc}
\usepackage{graphicx}
\usepackage{grffile}
\usepackage{longtable}
\usepackage{wrapfig}
\usepackage{rotating}
\usepackage[normalem]{ulem}
\usepackage{amsmath}
\usepackage{textcomp}
\usepackage{amssymb}
\usepackage{capt-of}
\usepackage{hyperref}
\usepackage{titling} \posttitle{\par\end{center}} \setlength{\droptitle}{-30pt} \usepackage{multicol} \setlength{\columnsep}{1cm} \usepackage[T1]{fontenc} \usepackage[utf8]{inputenc} \renewcommand{\contentsname}{Table of Contents / Agenda} \usepackage[letterpaper,left=1in,right=1in,top=0.7in,bottom=1in,headheight=23pt,includehead,includefoot,heightrounded]{geometry} \usepackage{fancyhdr} \pagestyle{fancy} \fancyhf{} \cfoot{\thepage} \usepackage{mathpazo} \usepackage[scaled=0.85]{helvet} \usepackage{courier} \usepackage[onehalfspacing]{setspace} \usepackage[framemethod=default]{mdframed} \usepackage{wrapfig} \usepackage{booktabs} \usepackage[outputdir=Lectures]{minted}
\setcounter{secnumdepth}{3}
\date{\vspace{-6ex}}
\title{Class 1: Introduction to Python: Devel Environments and Language Basics}
\hypersetup{
 pdfauthor={},
 pdftitle={Class 1: Introduction to Python: Devel Environments and Language Basics},
 pdfkeywords={},
 pdfsubject={Description School specific teaching materials},
 pdfcreator={Emacs 26.1 (Org mode 9.1.13)}, 
 pdflang={English}}
\begin{document}

\maketitle
\lhead{ COURSE 0000 \\ Joon H. Ro } 
\rhead{ Class 1 \\ 2018-08-28 Tue} 
\thispagestyle{fancy}

\setcounter{tocdepth}{1}
\tableofcontents
\vspace{6ex}

\section{Python Development Environment}
\label{sec:org00fe63a}
\begin{itemize}
\item In Data Science, usually your workflow is interactive
\item You need a text editor to write code and a shell to interactively run the
code
\item You can either have them separately, or use an IDE
\item I will only introduce cross-platform tools
\end{itemize}
\subsection{Python Installation}
\label{sec:org3b6026b}
\begin{itemize}
\item Python is not domain-specific language

\item One needs to install scientific libraries (SciPy Stack)
\end{itemize}

\begin{itemize}
\item Vanilla Python distribution from the official python website does not include
essential scientific libraries

\item It is highly recommended to use one of the scientific Python distributions
such as Anaconda
\end{itemize}

\subsubsection{Anaconda Scientific Python Distribution}
\label{sec:org676e46e}
\begin{itemize}
\item \url{https://www.anaconda.com/}
\item Install many scientific libraries at once
\item Manage libraries (install, update) conveniently (Anaconda Navigator)
\item Comes with Intel's MKL by default
\end{itemize}
\subsection{Text Editors and IDEs}
\label{sec:org536b148}
\begin{itemize}
\item When you code, you work with text
\item You will spend a lot of time with your text editor
\end{itemize}
\subsubsection{Text Editing Functionalities}
\label{sec:orgee2ea93}
\begin{itemize}
\item Syntax Highlighting
\item Column Editing
\item Search/Replace

\begin{itemize}
\item RegEx Search/Replace
\end{itemize}
\end{itemize}
\subsubsection{Examples}
\label{sec:orgd988253}
\begin{multicols}{2}
\begin{itemize}
\item Notepad, Notepad++
\item TextMate
\end{itemize}
\end{multicols}

\subsubsection{Vim and Emacs}
\label{sec:org4e0d949}
\begin{itemize}
\item Vi and Vim
\item Emacs
\item \href{http://en.wikipedia.org/wiki/Editor\_war\#Humor}{Editor war}
\item \href{http://www.slate.com/articles/technology/bitwise/2014/05/oldest\_software\_rivalry\_emacs\_and\_vi\_two\_text\_editors\_used\_by\_programmers.html}{The Oldest Rivalry in Computing}
\end{itemize}
\subsubsection{SublimeText}
\label{sec:org713cb1c}
\begin{itemize}
\item Multi selection

\begin{itemize}
\item Suppose I want to change 

\begin{minted}[]{python}
    X1, X2, X3, X4 = X1, X2, X3, X4
\end{minted}

\item To:

\begin{minted}[]{python}
    X1, X2, X3, X4 = data.X1, data.X2, data.X3, data.X4
\end{minted}
\end{itemize}

\item Dynamic setting application
\item Extensibility
\end{itemize}
\subsubsection{Current Recommendation}
\label{sec:org08fcfac}
\begin{itemize}
\item New generation text editors: SublimeText, Atom.io, Visual Studio Code
\item Current recommendation: Visual Studio Code
\begin{itemize}
\item Free software and Open Source
\item Powerful
\item Better performance than Atom.io
\item Many extensions (Python tooltip example)
\end{itemize}
\end{itemize}

\subsubsection{IDE Functionalities}
\label{sec:orgfb3fa97}
\begin{multicols}{2}
\begin{itemize}
\item Code execution

\begin{itemize}
\item Cell support
\end{itemize}

\item Debugging
\item Code checking
\item Project Management
\item Version control integration
\end{itemize}

\end{multicols}
\subsubsection{IDEs for Python}
\label{sec:org8940a23}
\begin{itemize}
\item Usually runs a Python interpreter within the application
\item Tight integration Editor and interpreter

\begin{itemize}
\item Advantage at debugging
\end{itemize}

\item Some candidates

\begin{itemize}
\item Spyder
\item PyCharm
\item WingIDE
\item PyScripter (Windows only)
\end{itemize}
\end{itemize}

\subsection{Code Snippets Manager}
\label{sec:org0021d05}
\begin{itemize}
\item Code Reuse 

\begin{itemize}
\item don't repeat yourself (DRY) principle
\item c.f. WET solutions: "write everything twice", "we enjoy typing" or "waste
everyone's time"
\end{itemize}

\item You want to accumulate frequently used code snippets for productivity
\item Current recommendation: Lepton (Gistbox became not-free)
\item Or you can use simpler things such as Simplenote
\end{itemize}
\section{IPython and Jupyter}
\label{sec:org9a3f91d}
\subsection{IPython and Jupyter}
\label{sec:org3fa5aef}
\begin{itemize}
\item IPython: Enhanced Python shell. Mainly, it provides

\begin{itemize}
\item tab completion
\item history search
\item on-the-fly documentation
\item \texttt{\%magic} functions
\item inline plotting
\end{itemize}
\end{itemize}

\begin{itemize}
\item Documents: \url{http://ipython.readthedocs.io/en/stable/interactive/tutorial.html}
\end{itemize}
\begin{itemize}
\item Jupyter
\begin{itemize}
\item IPython used to be used to specify both kernel and frontend (IPython,
IPython QT Console, IPython Notebook)
\item The frontend part became a language-agonastic separate project
\begin{itemize}
\item e.g., can use Julia and R kernel
\end{itemize}
\item Now it is Jupyter, which runs an IPython kernel by default
\end{itemize}
\end{itemize}

\begin{itemize}
\item One kernel, multiple frontend:
\begin{itemize}
\item QT Console
\item Notebook
\item Lab
\end{itemize}
\end{itemize}
\subsection{Jupyter Lab}
\label{sec:org2ed6766}
\begin{itemize}
\item The latest Jupyter frontend: Jupyter Lab
\item It is a flexible frontend which can encompass both Notebook and QT Console
\item We will use this throughout the semester for class
\end{itemize}

\subsubsection{Running Jupyter Lab}
\label{sec:orgb3640c1}
\begin{itemize}
\item Run \texttt{Jupyter Lab} from \texttt{Anaconda Navigator}
\begin{itemize}
\item Optionally you can create a shortcut to \texttt{jupyter lab}
\end{itemize}
\item You can run \texttt{Anaconda Prompt} and type \texttt{jupyter lab} \texttt{<Enter>}
\end{itemize}
\subsubsection{Creating a Notebook}
\label{sec:org5340012}
\begin{itemize}
\item Create a Notebook
\item Notebook consists of multiple cells which can be used for code or other things
\item You can insert a new cell with \texttt{Insert} menu
\item You can run a cell by pressing \texttt{Shift+Enter}
\end{itemize}

\begin{itemize}
\item Input the following in the first block: 

\begin{minted}[]{python}
  import pandas as pd
  pd.__version__
\end{minted}

\item Press \texttt{Shift+Enter} to run the code in the cell.
\end{itemize}
\subsubsection{Creating a Console for the Notebook}
\label{sec:org1d554c8}
\begin{itemize}
\item Right-click on a cell, select \texttt{Create Console for Notebook}.
\begin{itemize}
\item You can have a notebook and a console side-by-side in a browser tab.
\end{itemize}
\item You can rearrange the window layout
\item Remember the both notebook and console share the same Python kernel!
\item Press \texttt{Shift+Enter} to run the code (may change)
\begin{itemize}
\item If you want to change the behavior, see a discussion item on Canvas
\end{itemize}
\end{itemize}
\subsubsection{Can Open Text and Data (CSV) Files}
\label{sec:orgfb644c8}
\subsubsection{Workflow - Notebook + Console}
\label{sec:orgf04dc7e}
\begin{itemize}
\item Notebook and Qt Console are standalone programs
\item Throughout the semester, we will use Jupyter Lab for clarity
\end{itemize}
\subsection{Convenient Functionalities}
\label{sec:org7ca76d7}
\subsubsection{tab completion}
\label{sec:org94945ea}

\iffalse
\url{https://foxdeploy.files.wordpress.com/2017/01/upgrade-your-code1.png}
\fi

\begin{itemize}
\item The single most convenient functionality
\item With a partially completed expression, pressing \texttt{TAB} key either completes
the expression (when there is an unique expression available) or show candidates

\begin{minted}[]{python}
  >>> pr[TAB]

\end{minted}
\end{itemize}

\subsubsection{history search}
\label{sec:org6afcb82}
\begin{itemize}
\item In a console, you can browse the history of commands by \texttt{UP} and \texttt{DOWN} keys:

\begin{minted}[]{python}
  >>> [UP]
\end{minted}
\end{itemize}
\subsubsection{On-the-fly documentation}
\label{sec:orgf42edb4}
\begin{itemize}
\item If you press \texttt{Shift+Tab}, it will display documentation about the object
under the cursor
\item You can put \texttt{?} after an object and it will print out documentation
\end{itemize}
\subsubsection{\texttt{\%magic} functions}
\label{sec:org316c866}
\begin{itemize}
\item IPython provides many convenient magic functions.
\item \texttt{\%cd}: change working directory
\item \texttt{\%hist}: see history
\item \texttt{\%load}: load a Python script. Test it with an example from    
\url{http://matplotlib.org/gallery.html\#pie\_and\_polar\_charts}

\begin{itemize}
\item For example,
\end{itemize}
\begin{minted}[]{python}
    >>> %load http://matplotlib.org/mpl_examples/pie_and_polar_charts/polar_bar_demo.py
\end{minted}
\end{itemize}
\subsubsection{Inline plotting}
\label{sec:org05306c3}
\begin{itemize}
\item One of the most useful things is that it can show plots inline. Once you run
the following magic in Jupyter:

\begin{minted}[]{python}
  >>> %matplotlib inline
\end{minted}

\item Plots will be rendered inline. For example, run a cell with the following:

\begin{minted}[]{python}
  %load http://matplotlib.org/mpl_examples/pie_and_polar_charts/polar_bar_demo.py
\end{minted}

\item This makes the notebook very useful for interactive data exploration.
\item You can use \texttt{Create New View for Output} as well
\end{itemize}
\subsection{Jupyter QT Console Demo}
\label{sec:org13655cd}
\subsubsection{tab completion}
\label{sec:orgadd2f3b}
\subsubsection{history search}
\label{sec:org46e104c}
\subsubsection{on-the-fly documentation}
\label{sec:org3fed3c1}
\subsubsection{\texttt{\%magic} functions}
\label{sec:org868002b}

\begin{itemize}
\item You can run a script with \%run
\item You can load a script from the web with \%load:

\begin{minted}[]{python}
  >>> %load http://matplotlib.org/mpl_examples/pie_and_polar_charts/polar_bar_demo.py
\end{minted}

\item One of the most useful things is that it can show plots inline. You can run
the following magic:

\begin{minted}[]{python}
  >>> %matplotlib inline
\end{minted}

Then plots will be rendered inline. For example, run the following:

\begin{minted}[]{python}
  >>> %load http://matplotlib.org/mpl_examples/pie_and_polar_charts/polar_bar_demo.py
\end{minted}
\end{itemize}
\subsubsection{inline plotting}
\label{sec:orgabf96bb}
\section{Python Basics}
\label{sec:org04ad9ce}
\subsection{Basic Syntax}
\label{sec:org513b632}
\begin{itemize}
\item \texttt{=} is used for assignment:

\begin{minted}[]{python}
  >>> a = 10  # assign the value 10 to a variable named "a"
\end{minted}
\end{itemize}

\begin{itemize}
\item Python syntax is case-sensitive

\begin{minted}[]{python}
  >>> a  # give me a
  >>> A  # A does not exist
\end{minted}
\end{itemize}

\begin{itemize}
\item Pretty much anything (even unicode in Python 3) can be a variable name

\iffalse
\begin{minted}[]{python}
  >>> α = 10
  >>> α
\end{minted}
\fi

\texttt{>>>} \(\alpha\) \texttt{= 10}

\texttt{>>>} \(\alpha\)
\end{itemize}

\begin{itemize}
\item No need for a statement terminator (e.g., \texttt{;}). \texttt{;} is used to supress the
value of the last expression. (Mainly for interactive workflow)

\begin{minted}[]{python}
  >>> a
  >>> a;
\end{minted}
\end{itemize}

\begin{itemize}
\item \texttt{\#} is used for comments:

\begin{minted}[]{python}
  >>> print(10)  # this is a comment and will be ignored
\end{minted}
\end{itemize}

\begin{itemize}
\item Function calls always need parentheses, even when there is no argument:

\begin{minted}[]{python}
  >>> print("Hello World!")  # calling print function with argument "Hellow World!"
  >>> print()  # calling print function without any argument
  >>> print  # shows you the information about the function
\end{minted}
\end{itemize}

\subsection{Basic Data Types}
\label{sec:orgc3414bf}
\begin{itemize}
\item You can assign some value to a variable with \texttt{=}:

\begin{minted}[]{python}
  number = 1
\end{minted}

\begin{itemize}
\item type \texttt{number} to verify the value
\item You can use \texttt{type()} function to inspect an variable's type
\end{itemize}
\end{itemize}

\begin{itemize}
\item There are several types of data. The most basic ones are integer, float, and string:

\begin{minted}[]{python}
number_int = 1
number_float = 1.0
string = "My name is Joon"
\end{minted}
\end{itemize}

\subsubsection{String}
\label{sec:org1ed05ca}
\begin{itemize}
\item A string is usually a bit of text
\item You can use \texttt{"} and \texttt{'} interchangeably for strings
\begin{itemize}
\item Useful when you actually have quotes in a string. For example, if the
string you want to represents is \texttt{"This is an example string"}, then you
can use single quotes:

\begin{minted}[]{python}
    string = '"This is an example string"'
\end{minted}
\end{itemize}
\end{itemize}

\begin{itemize}
\item You can easily concatenate strings with \texttt{+} operator:

\begin{minted}[]{python}
  string = "My name is"
  print(string + ' ' + 'Joon Ro')
\end{minted}

\item Python's string provides a very useful string formatting functionality. If interested, see \url{https://docs.python.org/3.6/library/string.html}
\end{itemize}

\subsubsection{Built-in Constants}
\label{sec:orgbdeed61}
\begin{itemize}
\item There are more, but the most frequently used are:
\end{itemize}



\begin{description}
\item[{\texttt{False}}] The false value of the bool type. Assignments to False are
illegal and raise a SyntaxError.

\item[{\texttt{True}}] The true value of the bool type. Assignments to True are illegal
and raise a SyntaxError.

\item[{\texttt{None}}] The sole value of the type NoneType. None is frequently used to
represent the absence of a value
\end{description}
\subsection{Lists, Tuples, and Dictionaries}
\label{sec:orga1e3edc}
\begin{itemize}
\item In addition to the basic data types, there are many data types in
Python. e.g., lists, dictionaries, arrays, etc
\end{itemize}

\subsubsection{Lists}
\label{sec:org622d57e}
\begin{itemize}
\item Lists are one of the basic data types, and it is specified with \texttt{[]}
\item It can hold pretty much anything
\item For example:

\begin{minted}[]{python}
  >>> list_example = [1, 2, 'Third', 4, 'Fifth']
\end{minted}
\end{itemize}

\begin{itemize}
\item In general, you can use \texttt{len()} function to get the length of a data:

\begin{minted}[]{python}
  >>> len(list_example)
\end{minted}
\end{itemize}

\begin{itemize}
\item You always use integer index to access specific value(s) of a list
\item In Python, index starts with \texttt{0}:

\begin{minted}[]{python}
  >>> list_example[0]  # the first element
  >>> list_example[5]  # will give you an error since the last element is 4
\end{minted}
\end{itemize}
\subsubsection{Tuples}
\label{sec:orgf3a91d0}
\begin{itemize}
\item Similar to lists, but tuples are \emph{immutable}:

\begin{minted}[]{python}
  >>> tuple_example = (1, 2, 'Third', 4, 'Fifth')
\end{minted}

\item Accessing values is the same as lists
\item However, you cannot change values
\item Again, you can use \texttt{len()} to get the length of a tuple
\end{itemize}
\subsubsection{Dictionaries}
\label{sec:org729d7b3}
\begin{itemize}
\item You use a dictionary when you want to index an element with a meaningful thing instead of 
an integer:

\begin{minted}[]{python}
  dict_example = {}
  dict_example['name'] = 'Joon Ro'
\end{minted}

\item You can create it like this as well:

\begin{minted}[]{python}
  dict_example = {'name': 'Joon Ro',
                  'major': 'Marketing'}
\end{minted}
\end{itemize}
\subsection{Code Blocks in Python}
\label{sec:orgad1e7f4}
\begin{itemize}
\item In many cases, you have to specify multiple lines of code as a \emph{code block}
\item Note that in Python, blocks are distinguished by \emph{spaces}

\begin{itemize}
\item It forces you to indent, which improves readability of code a lot
\end{itemize}

\item For example,

\begin{minted}[]{python}
  if condition is True:
      print("I'm inside the if block")
      # do something
  
  print("I'm outside of if block")
\end{minted}
\end{itemize}
\subsubsection{Importance of indentation}
\label{sec:org97719b9}
\begin{minted}[]{c}
/*  Warning:  bogus C code!  */

if (some condition)
        if (another condition)
                do_something(fancy);
else
        this_sucks(badluck);
\end{minted}

\begin{itemize}
\item Either the indentation is wrong, or the program is buggy, because an "else" always applies to the nearest "if", unless you use braces. (Source: \url{http://www.secnetix.de/olli/Python/block\_indentation.hawk})
\end{itemize}
\subsubsection{Readability}
\label{sec:org7ccc5d1}
\begin{itemize}
\item Code is read much more often than it is written
\item You will NOT understand the code you wrote before!

\item Make sure to:

\begin{enumerate}
\item Comment your code appropriately
\item Use meaningful variable names
\item Indent nested code blocks properly
\end{enumerate}
\end{itemize}

\subsubsection{Tab VS. Spaces}
\label{sec:org91b19b8}
\begin{itemize}
\item Do not mix tab and spaces
\item Using 4 spaces for a tab is recommended
\end{itemize}
\subsection{Conditional Statements and Loops}
\label{sec:org3273d4f}
\begin{itemize}
\item Conditional statements and loops are what makes the automation possible
\item e.g., loop over each observation in the dataset, and do some calculation
depending on whether a variable value satisfies a condition
\end{itemize}
\subsubsection{Conditional Expressions}
\label{sec:org6e40cc0}
\begin{center}
\begin{tabular}{lll}
Meaning & Math Symbol & Python Symbols\\
\hline
Less than & < & \texttt{<}\\
Greater than & > & \texttt{>}\\
Less than or equal & ≤ & \texttt{<=}\\
Greater than or equal & ≥ & \texttt{>=}\\
Equals & = & \texttt{==}\\
Not equal & ≠ & \texttt{!=}\\
\end{tabular}
\end{center}

\subsubsection{\texttt{if} .. \texttt{elif} .. \texttt{else}}
\label{sec:org00e9409}
\begin{itemize}
\item \texttt{if} and \texttt{elif} will evaluate if the following conditional is \texttt{True}. If it
is, then it will evaluate the code block associated with it. Otherwise, it
will move to the next \texttt{elif}, or \texttt{else}, or out of the \texttt{if} statement

\begin{minted}[]{python}
  if condition is True:
      print("I'm inside the if block")
      # do something
  
  print("I'm outside of if block")
\end{minted}
\end{itemize}

\begin{minted}[]{python}
a = 10
b = 5

if a > b:
   print("a > b")

elif a < b:  # will not be evaluated if the above condition is true
   print("a < b")

else:  # will not be evaluated if any of the the above conditions is true
   print("a == b")
\end{minted}

\begin{itemize}
\item You can just use a number for the condition in the \texttt{if} statements

\begin{itemize}
\item \texttt{0} is like \texttt{False}. Any number other than \texttt{0} will be regarded as \texttt{True}

\begin{minted}[]{python}
    if True:
       print("I will always run")
    
    if 0:
       print("I will never run")
\end{minted}
\end{itemize}
\end{itemize}
\subsubsection{\texttt{for} loop}
\label{sec:org1fb886a}
\begin{itemize}
\item \texttt{for} loop will loop over an iterable object and apply the operation inside the block 
to each element of the object:
\end{itemize}

\begin{minted}[]{python}
for counter in (an iterable):
    print("I'm inside the for block")
    # do something

print("I'm outside of for block")
\end{minted}

\begin{itemize}
\item An iterable object is usually a list (but anything can be used)
\begin{minted}[]{python}
  for number in [1, 2, 3]:  # number will take value 1, 2, 3
      another_number = number + 3  # going to be 4, 5, 6
      print(another_number)
\end{minted}
\end{itemize}

\begin{itemize}
\item An useful built-in function: \texttt{range()}, which gives you a range of numbers
\begin{minted}[]{python}
  list_numbers_from_range = range(10)  # 10 numbers: 0, 1, 2, ... , 9
  
  for number in list_numbers_from_range:
      print(number)
\end{minted}
\end{itemize}

\begin{itemize}
\item Often we want to count numbers. For example,

\begin{minted}[]{python}
  list_numbers_from_range = range(10)  # 0, 1, 2, ... , 9
  
  i = 0  # initialize the counter
  for number in list_numbers_from_range:
      i = i + 1  # equivalently, i += 1

  print(i)
\end{minted}
\end{itemize}
\subsubsection{Control \texttt{for} loop with conditional breaking and continuation}
\label{sec:orge66a623}
\begin{itemize}
\item You can \texttt{break} a \texttt{for} loop with break:

\begin{minted}[]{python}
  for number in range(10):
      if number > 5:
          break

  print(number)
\end{minted}
\end{itemize}

\begin{itemize}
\item You can also skip one run of the loop with \texttt{continue}:

\begin{minted}[]{python}
  sum_numbers = 0

  for number in range(10):
      if number > 5:
          continue  # will skip all statements below within the block

      sum_numbers += number

  print(number)
\end{minted}
\end{itemize}
\subsubsection{Simple debugging by raising an exception}
\label{sec:orgc926013}
\begin{itemize}
\item Remember that all the variables will retain their values when the loop stops.
\item You can do a simple debugging by forcing an exception:

\begin{minted}[]{python}
  for number in range(10):
      if number > 5:
          1 / 0
\end{minted}
\end{itemize}
\end{document}