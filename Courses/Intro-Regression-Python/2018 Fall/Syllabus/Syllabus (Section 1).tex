% Intended LaTeX compiler: pdflatex
\documentclass[article,letterpaper,times,10pt,listings-bw,microtype]{scrartcl}
\usepackage[utf8]{inputenc}
\usepackage[T1]{fontenc}
\usepackage{graphicx}
\usepackage{grffile}
\usepackage{longtable}
\usepackage{wrapfig}
\usepackage{rotating}
\usepackage[normalem]{ulem}
\usepackage{amsmath}
\usepackage{textcomp}
\usepackage{amssymb}
\usepackage{capt-of}
\usepackage{hyperref}
\usepackage[onehalfspacing]{setspace} \usepackage[T1]{fontenc} \usepackage{mathpazo} \usepackage[scaled=0.85]{helvet} \usepackage{courier} \usepackage{geometry} \geometry{left=1in,right=1in,top=1in,bottom=1in} \usepackage[framemethod=default]{mdframed}
\author{Your Name Your Name}
\date{Fall 2018}
\title{COURSE 0000-01 Syllabus}
\hypersetup{
 pdfauthor={Your Name Your Name},
 pdftitle={COURSE 0000-01 Syllabus},
 pdfkeywords={},
 pdfsubject={Description School specific teaching materials},
 pdfcreator={Emacs 26.1 (Org mode 9.1.13)}, 
 pdflang={English}}
\begin{document}

\definecolor{SchoolColor}{RGB}{70,130,180}

\begin{center}
{\color{SchoolColor}{\Large
\textbf{School Name}
}}
\end{center}

\begin{center}
{\color{SchoolColor}{
\textbf{COURSE 0000-01}

\textbf{Introduction to Regression Analysis with Python}

\textbf{Fall 2018}
}}
\end{center}

\vspace{5 mm}

\begin{center}
\begin{tabular}{llll}
\textbf{Instructor:} & Joon H. Ro & \textbf{Office Phone:} & (000) 000-0000\\
\textbf{Office:} & BLDG 100 & \textbf{E-mail:} & joon.ro@outlook.com\\
\textbf{Office Hours:} & Tue 3:30-4:30pm & \textbf{Course Site:} & \textbf{\href{https://github.com}{github.com}}\\
\textbf{Class Meeting Day \& Time:} & Tue/Thurs, 9:30a-10:45 & \textbf{Class Location:} & BLDG 100\\
\end{tabular}
\end{center}
\section*{Course Description}
\label{sec:org2148b25}
This course provides students with Org-Coursepack, a modular and reusable
teaching materials template in org-mode. Using self-explanatory course
contents, students will be able to adapt the template and create their on
course contents through Org-mode.
\section*{Course Prerequisites}
\label{sec:org66d41c3}
\begin{itemize}
\item Basic Statistics knowledge
\end{itemize}
\section*{Student Learning Objectives}
\label{sec:orga46d8fc}
As the result of this course, students should be able to:

\begin{itemize}
\item Interpret regression results
\item Understand how to implement elasticity models in regression
\item Run regression analysis in Python
\end{itemize}
\section*{Course Material}
\label{sec:org995f75f}
\begin{description}
\item[{Documentation}] Handouts, readings, and assignments will be uploaded to
the repository.
\item[{Software}] Most data manipulations and analyses will be done using Python

\begin{mdframed}[style=exampledefault, frametitle={Note}]
\begin{itemize}
\item Please install Anaconda (can be downloaded from
\url{https://www.anaconda.com/download/}). Make sure you install Python 3.6
version (64-bit version is recommended)
\item Please bring your laptop for each class for in-class exercises
\end{itemize}
\end{mdframed}
\end{description}
\clearpage
\section*{Class Schedule}
\label{sec:orgdf2b3c4}
\begin{center}
\begin{tabular}{lrl}
Date & Class & Topic\\
\hline
2018-08-28 Tue & 1 & Introduction to Python: Devel Environments and Language Basics\\
2018-08-30 Thu & 2 & Regression Analysis: Introduction\\
2018-09-04 Tue & 3 & Hypothesis Testing for a Mean and Significance of Regression Coefficients\\
2018-09-06 Thu & 4 & Multiple Regression and Categorical Variables\\
2018-09-11 Tue & 5 & Design of Price and Advertising Elasticity Models\\
2018-09-13 Thu & 6 & Interaction Effects and Overfitting\\
\end{tabular}
\end{center}

\begin{mdframed}[style=exampledefault, frametitle={Disclaimer}]
\begin{itemize}
\item The class schedule is subject to change (except for the exam dates)
\end{itemize}
\end{mdframed}
\end{document}