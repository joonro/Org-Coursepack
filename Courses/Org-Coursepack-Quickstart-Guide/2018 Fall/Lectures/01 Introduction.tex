% Intended LaTeX compiler: pdflatex
\documentclass[10pt,article]{article}
\usepackage[utf8]{inputenc}
\usepackage[T1]{fontenc}
\usepackage{graphicx}
\usepackage{grffile}
\usepackage{longtable}
\usepackage{wrapfig}
\usepackage{rotating}
\usepackage[normalem]{ulem}
\usepackage{amsmath}
\usepackage{textcomp}
\usepackage{amssymb}
\usepackage{capt-of}
\usepackage{hyperref}
\usepackage{titling} \posttitle{\par\end{center}} \setlength{\droptitle}{-30pt} \usepackage{multicol} \setlength{\columnsep}{1cm} \usepackage[T1]{fontenc} \usepackage[utf8]{inputenc} \renewcommand{\contentsname}{Table of Contents / Agenda} \usepackage[letterpaper,left=1in,right=1in,top=0.7in,bottom=1in,headheight=23pt,includehead,includefoot,heightrounded]{geometry} \usepackage{fancyhdr} \pagestyle{fancy} \fancyhf{} \cfoot{\thepage} \usepackage{mathpazo} \usepackage[scaled=0.85]{helvet} \usepackage{courier} \usepackage[onehalfspacing]{setspace} \usepackage[framemethod=default]{mdframed} \usepackage{wrapfig} \usepackage{booktabs} \usepackage[outputdir=Lectures]{minted}
\setcounter{secnumdepth}{3}
\date{\vspace{-6ex}}
\title{Class 1: Introduction}
\hypersetup{
 pdfauthor={},
 pdftitle={Class 1: Introduction},
 pdfkeywords={},
 pdfsubject={Org-Coursepack quickstart guide School specific teaching materials},
 pdfcreator={Emacs 26.1 (Org mode 9.1.14)}, 
 pdflang={English}}
\begin{document}

\maketitle
\lhead{ COURSE 0000 \\ Your Name } 
\rhead{ Class 1 \\ 2018-08-28 Tue} 
\thispagestyle{fancy}

\setcounter{tocdepth}{1}
\tableofcontents
\vspace{6ex}

\section{Introduction to Org-Coursepack Quickstart guide}
\label{sec:org4a2b6c9}
Content for this section here.
\section{New section}
\label{sec:orgc9353ff}
\subsection{Lists}
\label{sec:org9ba10fb}
Obviously you cannot use \texttt{*} to specify a list, but otherwise Org mode
uses a typical syntax for lists. For example,

\begin{itemize}
\item List
\begin{itemize}
\item Nested list item 1
\item Nested list item 2
\end{itemize}
\end{itemize}


\begin{enumerate}
\item Numbered list
\end{enumerate}

\subsection{Math}
\label{sec:org88a9dc3}
You can directly input \LaTeX{} math in Org mode. For example,

\[ \cos (2\theta) = \cos^2 \theta - \sin^2 \theta \]
\subsection{Slide split}
\label{sec:org6a4cbce}
Users can put \texttt{\#+REVEAL: split} to split a slide. For example,

This line will be shown in a new slide.
\subsection{Fragmented Contents}
\label{sec:org0a59814}
\begin{itemize}
\item This list
\item is fragmented
\item in reveal.js slides
\end{itemize}
\subsection{Local Images}
\label{sec:org42ec809}
Local images can be included in this way:

\begin{center}
\includegraphics[width=4cm]{../../../Assets/Images/Misc/affiche.png}
\end{center}

\subsection{Web Images / Hiding Contents}
\label{sec:orgd6f9897}
Remote images can be included by directly using its URL. 

\iffalse
\url{https://openclipart.org/image/400px/svg\_to\_png/125899/affiche.png}

(This image is not shown in \LaTeX{} output.)
\fi

However, it only works for HTML output - for \LaTeX{} an local image is needed.

(This sentence is not shown in HTML output.)
\end{document}